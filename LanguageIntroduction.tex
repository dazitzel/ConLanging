% Based on a Template which is
%     Creative Commons Attribution-ShareAlike 4.0 International
%     (CC BY-SA 4.0)
%     https://creativecommons.org/licenses/by-sa/4.0/

\chapter{Introduction to the Template}

\begin{center}
\begin{tabular}{|p{0.7\textwidth}|}
\hline
I want to thank LangTimeStudio for their \textsl{Language Creation Template}.
My first version of this was just a redress of that.
I'm to the point where I believe this to now be a separate document --- though I do still
plagiarize large sections of the original.
I have not received any endorsement from them.
\\\hline
\end{tabular}

\vskip 5mm

\begin{tabular}{|p{0.7\textwidth}|}
\hline
This chapter could be missing from your final product.
That doesn't mean that you shouldn't have an introduction, if you think it would help.
It just means that it isn't \emph{strictly} needed for creating a language.
\\\hline
\end{tabular}
\vskip 5mm

\begin{tabular}{|p{0.7\textwidth}|}
\hline
Finally, while the \texttt{pdf} version of this document can serve as a template and be
copy/pasted into a word processing document, there is quite a bit more value in using the
\TeX\ source and editing that instead.
\\\hline
\end{tabular}
\end{center}

Hopfully the prior paragraphs are an obvious enough meta-comment in this template.
The point of this document is to show how a constructed language could be documented.
It tends to have as much as I can put into a section, with comments (like the prior paragraph)
about how in-depth your constructed language is.

If you do a search on \texttt{conlang.org} you will find a list of language levels and suggested
prices.
I don't list any prices here.
If you are looking into being a paid conlanger you need to double-check what the going rate is.

Before we get to actual language creation, we need to talk about the bug-a-boo of what some people
think they are asking for when they ask for a constructed language; sadly this is even true in
professional and very successful film franchises.

\section{Vector Drawings}

The goal here is to create artwork, for a grand total of somewhere between 25 and 30 glyphs.
Generally because what we are doing here is actually creating a different way to write English.
The reason we say between 25 and 30 rather than 26 is because we occasionally an additional symbol
for /ʃ/ and /θ/ or /ð/ and are perhaps throwing away Q and X.

\begin{center}
\begin{tabular}{|cp{0.7\textwidth}|}
\hline
\multicolumn{2}{|l|}{What you provide:}\\
\textbullet&Digital vector drawings of each individual glyph\\
\textbullet&1 sentence rendered in the original writing system\\
\hline
\end{tabular}
\end{center}

For an idea of how you might document a set of glyphs, check out the chapter called ``The Look of
\LanguageName'' and the section on ``Orthography''.

\section{Digital Font}

The goal here is to create a font that allows you to almost type English and look like a different
language.
It should have between 25 and 30 glyphs, though 26 is much more likely.
We occasionally find things like Q or X serving as symbols for /ʃ/ and /θ/ or /ð/ or expect the
typist to replace those letters with something more phonomic.
Of course, modern fonts allow for quite a bit of freedom in having those additional symbols.

\begin{center}
\begin{tabular}{|cp{0.7\textwidth}|}
\hline
\multicolumn{2}{|l|}{What you provide:}\\
\textbullet&An \texttt{otf} or \texttt{ttf} file containing the digital font package of your
original writing system with combination kerning for optimized spacing\\
\textbullet&Instructions for installing and using the font on your own computer\\
\textbullet&1 sentence rendered in the original writing system\\
\hline
\end{tabular}
\end{center}

For an idea of how you might document a font, check out the chapter called ``The Look of
\LanguageName'' and the sections on ``Orthography'' and ``Typing \LanguageName''.

\vskip 5 mm

Now onto the \emph{real} constructed languages.

\section{Naming Language}

A language created to provide a consistent naming structure for people and/or places.

\begin{center}
\begin{tabular}{|cp{0.7\textwidth}|}
\hline
\multicolumn{2}{|l|}{What you provide:}\\
\textbullet&Phonology (the sounds of the language)\\
\textbullet&24 created names\\
\hline
\end{tabular}
\end{center}

In general, neither drawings nor fonts will be made for a naming language.
Names tend to be written in a modern othography regardless of the source language of the names,
giving you things like Onondaga in New York while you find Tulalip in Washington.\footnote{I
\emph{am} aware that those aren't naming languages and are real languages. My point is that this is
how naming languages tend to be used within a fictional setting because of the way names are
borrowed in real places.}

\section{Conlang Sketch}

This is something more appropriate for a language used in limited circumstances; like official
decrees, magic incantations, or barked orders on a starship.

\begin{center}
\begin{tabular}{|cp{0.7\textwidth}|}
\hline
\multicolumn{2}{|l|}{What you provide:}\\
\textbullet&Phonology (the sounds of the language)\\
\textbullet&A simplified grammar that describes the order of words in a sentence (such as
subject-verb-object word order)\\
\textbullet&A dictionary of 50 words\\
\hline
\end{tabular}
\end{center}

One thing that may be wanted along with a conlang sketch is a unique way to write it.
Making a whole font is probably overkill for a limited language used in limited circumstances.
But if some digital vector drawings, along with one sentence fully rendered in images; then don't
forget to adjust your pricing accordingly --- though the price is not ``price of sketch plus price
of vector drawings.''
There should be some level of discount for doing both at the same time.

\section{Basic Conlang}

Enough of a language that you can actually make some dialogue.

\begin{center}
\begin{tabular}{|cp{0.7\textwidth}|}
\hline
\multicolumn{2}{|l|}{What you provide:}\\
\textbullet&Phonology (the sounds of the language)\\
\textbullet&A basic grammar sufficient to translate simple sentences\\
\textbullet&5 to 10 sample translations (one sentence each)\\
\textbullet&A dictionary of 150 words\\
\hline
\end{tabular}
\end{center}

This is just for the conlang, not any unique writing.
The first question to ask yourself is if you even want a unique writing system for it.
Perhaps literacy is low enough that it doesn't make sense, maybe it's a language that lacks
prestige.

If you do want to write it, then you must decide if you want images, which are easier to make but
harder to string into sentences; or a font, which is harder to make but easier to string into
sentences.
Either way you need to adjust your pricing accordingly.

\section{Advanced Grammar}

As much as possible, this template has tried to provide what is needed to make an advanced grammar.

\begin{center}
\begin{tabular}{|cp{0.7\textwidth}|}
\hline
\multicolumn{2}{|l|}{What you provide:}\\
\textbullet&Phonology (the sounds of the language)\\
\textbullet&A full grammar that includes descriptions of how to create questions, complex sentences
with multiple clauses, and advanced language features\\
\textbullet&20 to 30 sample translations (one sentence each)\\
\textbullet&A dictionary of 500 words\\
\hline
\end{tabular}
\end{center}

Just like with a basic conlang this is just for the conlang, not any unique writing.
The first question to ask yourself is if you even want a unique writing system for it.
Perhaps they haven't invented writing yet, maybe it's considered too sacred to write down.

If you do want to write it, then you will want to make a font and add a rendering of the ``5 to
10'' translations using the font and adjust your price accordingly.

\section{What Now?}

Now that we have explained the different levels, you will find boxes throughout this document just like at the beginning of this chapter that will point out things like this chapter isn't used in a naming language, or this section is only for an advanced grammar.

In each case I've tried to include a large sample of what can be, so that you can focus on being creative.

\vskip 1 cm

\rightline{Enjoy.}

