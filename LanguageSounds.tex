% Based on a Template which is
%     Creative Commons Attribution-ShareAlike 4.0 International
%     (CC BY-SA 4.0)
%     https://creativecommons.org/licenses/by-sa/4.0/

\chapter{The Sounds of \LanguageName}

\section{Consonants}

\LanguageName\ uses the following consonants.
In this abbreviated \textsc{ipa} table, any symbols on the left are unvoiced (like the most common sound of an f) and any symbols on the right are voiced (like a v).

\begin{center}
\newcommand{\head}{\fontsize{7pt}{7pt}\selectfont}
\begin{tabular}{|r|cc|cc|cccccc|cc|cc|cc|}\hline
&
\multicolumn{2}{|c|}{\head Bilabial}&
\multicolumn{2}{|c|}{\head Labiodental}&
\multicolumn{2}{|c|}{\head Dental}&
\multicolumn{2}{|c|}{\head Alveolar}&
\multicolumn{2}{|c|}{\head Postalveolar}&
\multicolumn{2}{|c|}{\head Palatal}&
\multicolumn{2}{|c|}{\head Velar}&
\multicolumn{2}{|c|}{\head Glottal}\\\hline
              {\head Nasal}& &m& &ɱ& & & &n& & & &ɳ& &ŋ&\cellcolor{gray}&\cellcolor{gray}\\\hline
              {\head Trill}& &ʙ& & & & & &r& & & & &\cellcolor{gray}&\cellcolor{gray}&\cellcolor{gray}&\cellcolor{gray}\\\hline
            {\head Plosive}&p&b& & & & &t&d& & &c&ɟ&k&g&ʔ&\cellcolor{gray}\\\hline
        {\head Affricative}& & & & & &\multicolumn{1}{c|}{}&\multicolumn{1}{|c}{ts}&\multicolumn{1}{c|}{dz}&\multicolumn{1}{|c}{tʃ}&dʒ& & & & & &\cellcolor{gray}\\\hline
          {\head Fricative}&ɸ&β&f&v&θ&\multicolumn{1}{c|}{ð}&\multicolumn{1}{|c}{s}&\multicolumn{1}{c|}{z}&\multicolumn{1}{|c}{ʃ}&ʒ&ç&ʝ&x&ɣ&h&ɦ\\\hline
        {\head Approximant}& & & &ʋ& & & &ɹ& & & &j& &w&\cellcolor{gray}&\cellcolor{gray}\\\hline
{\head Lateral Approximant}&\cellcolor{gray}&\cellcolor{gray}&\cellcolor{gray}&\cellcolor{gray}&&&&l&&&&&&ʟ&\cellcolor{gray}&\cellcolor{gray}\\\hline
\end{tabular}
\end{center}

For comparison, here are the consonants often heard in American English.

\begin{center}
\newcommand{\head}{\fontsize{7pt}{7pt}\selectfont}
\begin{tabular}{|r|cc|cc|cccccc|cc|cc|cc|}\hline
&
\multicolumn{2}{|c|}{\head Bilabial}&
\multicolumn{2}{|c|}{\head Labiodental}&
\multicolumn{2}{|c|}{\head Dental}&
\multicolumn{2}{|c|}{\head Alveolar}&
\multicolumn{2}{|c|}{\head Postalveolar}&
\multicolumn{2}{|c|}{\head Palatal}&
\multicolumn{2}{|c|}{\head Velar}&
\multicolumn{2}{|c|}{\head Glottal}\\\hline
              {\head Nasal}& &m& & & &k& &n& & & & & &ŋ&\cellcolor{gray}&\cellcolor{gray}\\\hline
              {\head Trill}& & & & & & & & & & & & &\cellcolor{gray}&\cellcolor{gray}&\cellcolor{gray}&\cellcolor{gray}\\\hline
            {\head Plosive}&p&b& & & & &t&d& & & & &k&g& &\cellcolor{gray}\\\hline
        {\head Affricative}& & & & & &\multicolumn{1}{c|}{}&\multicolumn{1}{|c}{}&\multicolumn{1}{c|}{}&\multicolumn{1}{|c}{tʃ}&dʒ& & & & & &\cellcolor{gray}\\\hline
          {\head Fricative}& & &f&v&θ&\multicolumn{1}{c|}{ð}&\multicolumn{1}{|c}{s}&\multicolumn{1}{c|}{z}&\multicolumn{1}{|c}{ʃ}& & & & & &h&\\\hline
        {\head Approximant}& & & & & & & &ɹ& & & &j& &w&\cellcolor{gray}&\cellcolor{gray}\\\hline
{\head Lateral Approximant}&\cellcolor{gray}&\cellcolor{gray}&\cellcolor{gray}&\cellcolor{gray}&&&&l&&&&&&ʟ&\cellcolor{gray}&\cellcolor{gray}\\\hline
\end{tabular}
\end{center}

\eject

\section{Vowels}

\LanguageName\ uses the following vowels.
Any symbols before a dot (\textbullet) are pronounced with unrounded lips and any symbols after a dot are pronounced with rounded lips.

\begin{center}
\begin{tikzpicture}[scale=0.9]
\draw[ultra thick](0,6)--(4,0);
\draw[ultra thick](3.6,6)--(5.5,0);
\draw[ultra thick](7,6)--(7,0);
\draw[ultra thick](0.7,6)--(3,6);
\draw[ultra thick](4.5,6)--(5.9,6);
\draw[ultra thick](2,4)--(3.5,4);
\draw[ultra thick](5,4)--(6.3,4);
\draw[ultra thick](3.8,2)--(4.2,2);
\draw[ultra thick](5.6,2)--(6.3,2);
\draw[ultra thick](4.9,0)--(6.2,0);
\fill[white](4.3,2.7)rectangle(4.8,3.4);
\fill[white](4.9,0.7)rectangle(5.5,1.4);
\draw(0,6)node{\huge i\raisebox{-1.3mm}{\Huge\textbullet}y};
\draw(3.7,6)node{\huge ɨ\raisebox{-1.3mm}{\Huge\textbullet}ʉ};
\draw(7,5.9)node{\huge ɯ\raisebox{-1.3mm}{\Huge\textbullet} u};
\draw(2,5)node{\huge ɪ\raisebox{-1.3mm}{\Huge\textbullet}ʏ};
\draw(5.5,5)node{\huge\ \raisebox{-1.3mm}{\Huge\textbullet}ʊ};
\draw(1.4,3.95)node{\huge e\raisebox{-1.3mm}{\Huge\textbullet}ø};
\draw(4.25,3.95)node{\huge ɘ\raisebox{-1.3mm}{\Huge\textbullet}ɵ};
\draw(7,3.9)node{\huge ɤ\raisebox{-1.3mm}{\Huge\textbullet}o};
\draw(4.5,3)node{\huge ə};
\draw(2.75,1.95)node{\huge ɛ\raisebox{-1.3mm}{\Huge\textbullet}œ};
\draw(4.9,1.95)node{\huge ɜ\raisebox{-1.3mm}{\Huge\textbullet}ɞ};
\draw(7,1.95)node{\huge ʌ\raisebox{-1.3mm}{\Huge\textbullet}ɔ};
\draw(3.1,1)node{\huge æ\raisebox{-1.3mm}{\Huge\textbullet}\ };
\draw(5.2,1)node{\huge ɐ};
\draw(4.05,-0.05)node{\huge a\raisebox{-1.3mm}{\Huge\textbullet}ɶ};
\draw(7,-0.05)node{\huge ɑ\raisebox{-1.3mm}{\Huge\textbullet}ɒ};
\draw(0,6.8)node{Front};
\draw(3.5,6.8)node{Central};
\draw(7,6.8)node{Back};
\draw(-3,6)node[right]{Close};
\draw(-3,4)node[right]{Close-mid};
\draw(-3,2)node[right]{Open-mid};
\draw(-3,0)node[right]{Open};
\end{tikzpicture}
\end{center}

For comparison, here are the vowels usually heard in American English.

\begin{center}
\begin{tikzpicture}[scale=0.9]
\draw[ultra thick](0,6)--(4,0);
\draw[ultra thick](3.6,6)--(5.5,0);
\draw[ultra thick](7,6)--(7,0);
\draw[ultra thick](0,6)--(7,6);
\draw[ultra thick](1.2,4)--(7,4);
\draw[ultra thick](2.7,2)--(7,2);
\draw[ultra thick](4,0)--(4.9,0);
\fill[white](4.3,2.7)rectangle(4.8,3.4);
\fill[white](5,0)rectangle(6,0.3);
\draw(-0.5,6)node{\huge seat\raisebox{-1.3mm}{\Huge\textbullet}};
\draw(7.5,6)node{\huge\raisebox{-1.3mm}{\Huge\textbullet}suit};
\draw(2,5)node{\huge sit\raisebox{-1.3mm}{\Huge\textbullet}};
\draw(5.7,5)node{\huge\raisebox{-1.3mm}{\Huge\textbullet}soot};
\draw(0.8,4)node{\huge sate\raisebox{-1.3mm}{\Huge\textbullet}};
\draw(7.7,4)node{\huge\raisebox{-1.3mm}{\Huge\textbullet}soak};
\draw(4.55,3)node{\huge sup};
\draw(2.3,1.95)node{\huge set\raisebox{-1.3mm}{\Huge\textbullet}};
\draw(3.6,0)node{\huge sat\raisebox{-1.3mm}{\Huge\textbullet}};
\draw(6.1,0)node{\huge sought\raisebox{-1.3mm}{\Huge\textbullet}};
\draw(0,6.8)node{Front};
\draw(3.5,6.8)node{Central};
\draw(7,6.8)node{Back};
\draw(-3,6)node[right]{Close};
\draw(-3,4)node[right]{Close-mid};
\draw(-3,2)node[right]{Open-mid};
\draw(-3,0)node[right]{Open};
\end{tikzpicture}
\end{center}

\section{Tone}

If the language has lexical tone, state whether it’s register or contour, then list either all the tones, or all the tone melodies.
Also make a note if tone is used grammatically.

\section{Multiple Vowels}

The following dipthongs/tripthongs are considered one vowel within a syllable: ai, ei, oi, ui.

Doubled vowels, or geminates, do occur in \LanguageName.
They are pronounced as separate syllables with a small pause between the vowels.
So \textbf{aa} is pronouced ``a'a'' and not just held longer.

\section{Double Consonants}

Doubled consonants, or geminates, do occur in \LanguageName.
To pronounce a doubled consonant, simply pronounce it twice.
You might think of it as lingering over the consonant.
Think of the ``\uline{s}'' sound you pronounce in ``Mi\uline{ss S}ally''.
It's a longer ``\uline{s}'' than if you pronounce the similar phrase ``Mi\uline{ss} Ally''.

\section{Syllable Structure}

In the most general form a syllable in \LanguageName\ is (s/ʃ)(C)(C)V(C)(C).

Trills and lateral approximants are grouped together as ``liquid'' and are most common in longer strings of consonants.

Onsets include:
any single consonant except /j/ before /i/;
Consonants + /j/ if the following vowel is not /i/;
fricative + nasal;
plosive + liquid;
plosive + nasal;
plosive + /v/;
voiceless sibilant (s/ʃ) + voiceless stop (+ liquid).

Codas include:
any single consonant except /h/;
nasal + voiced affricative;
nasal + voiced fricative;
plosive + liquid.

In compound words that might place un-allowed cononant clusters together, an emphatic /a/ is placed between the roots.

\section{Stress}

If the language has a regular stress pattern, state it here.
If it's used gramatically, explain it.
If it's irregular or could be considered an extra dimension to the vowel, say so.

