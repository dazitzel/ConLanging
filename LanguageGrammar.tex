% Based on a Template which is
%     Creative Commons Attribution-ShareAlike 4.0 International
%     (CC BY-SA 4.0)
%     https://creativecommons.org/licenses/by-sa/4.0/

\chapter{\LanguageName\ Grammar Sketch}

\begin{center}
\begin{tabular}{|p{0.7\textwidth}|}
\hline
This chapter, just after a detailed description, may seem supurfluous.
The point of the previous chapter was to provide a reference that is the final law.
The point of this chapter is to provide a cheat sheet.
\\\hline
\end{tabular}
\end{center}

\section{Pronouns}

\begin{center}
\begin{tabular}{|l|l|l|}\hline
&\textbf{Singular}&\textbf{Plural}\\\hline
\textbf{First Person}&A&B\\\hline
\textbf{Second Person}&C&D\\\hline
\textbf{Third Person (feminine)}&E&\\\cline{1-2}
\textbf{Third Person (inanimate)}&F&G\\\cline{1-2}
\textbf{Third Person (masculine)}&H&\\\hline
\textbf{Fourth Person}&\multicolumn{2}{c|}{I}\\\hline
\textbf{Reflexive}&\multicolumn{2}{c|}{J}\\\hline
\end{tabular}
\end{center}

\section{Nouns}

\textbf{word} \textit{``gloss''}

\begin{center}
\begin{tabular}{|l|l|l|}\hline
\textit{Inflection}&\textbf{Singular}&\textbf{Plural}\\\hline
\textbf{Nominative}&word&word\textbf{a}\\\hline
\textbf{Subjective}&word\textbf{b}&word\textbf{c}\\\hline
\end{tabular}

\begin{tabular}{|l|l|l|}\hline
\multicolumn{3}{|l|}{\textit{Derivation}}\\\hline
\textbf{Augment}&word\textbf{d}&``glossiest''\\\hline
\textbf{Dim}&word\textbf{e}&``glossless''\\\hline
\end{tabular}
\end{center}

\section{Adjectives}

\textbf{word} \textit{``gloss''}

\begin{center}
\begin{tabular}{|l|l|l|}\hline
&\textbf{Singular}&\textbf{Plural}\\\hline
\textbf{Nominative}&word&word\\\hline
\textbf{Subjective}&word&word\\\hline
\end{tabular}
\end{center}

\section{Verbs}

\textbf{word} \textit{``gloss''}, \textbf{worda} \textit{``gless''}, \textbf{wordb} \textit{``glass''}

\begin{center}
\begin{tabular}{|r|l|l|l|l|l|l|}\hline
\textbf{Infinive Mood}&\multicolumn{5}{|l|}{word\textbf{i}}&``to gloss/gless/glass'' \\\hline
\textbf{Past Tense}&word\textbf{is}&\textit{``glossed''}&word\textbf{ais}&\textit{``glessed''}&word\textbf{bis}&\textit{``glassed''}\\\hline
\end{tabular}
\end{center}

\section{Participles}

\begin{center}
\begin{tabular}{|p{0.7\textwidth}|}
\hline
If you are providing only a Basic Grammar, then this section should be missing.
After all, you didn't define them in the prior chapter, did you?
\\\hline
\end{tabular}
\end{center}

\textbf{word} ``to gloss''

\begin{center}
\begin{tabular}{|r|ll|rr|}\hline
&\multicolumn{2}{c|}{\textbf{Active}}&\multicolumn{2}{|c|}{\textbf{Passive}}\\\hline
\textbf{Incomplete}&word\textbf{a}a&``was glossing''&word\textbf{a}b&``was glossed''   \\\hline
\textbf{Complete}  &word\textbf{b}a&``glossed''     &word\textbf{b}b&``being glossed'' \\\hline
\textbf{Repeat}    &word\textbf{c}a&``glossing''    &word\textbf{d}b&``being glossing''\\\hline
\end{tabular}
\end{center}

\section{Numbers}

\begin{center}
\begin{tabular}{|p{0.7\textwidth}|}
\hline
I have listed glyphs here.
There are actually several possibilites.
One possibility is that glyphs don't exist --- they just use words.
Another possibility is that you haven't made vector images or a font and so have no glyhs.
It could also be that they just re-use their orthography --- perhaps with a small symbol before, after, or around it.
\\\hline
\end{tabular}
\end{center}

\begin{center}
\begin{tabular}{|c|c|l|c|c|l|c|c|l|}\hline
\textbf{\#}&\textbf{Glyphs}&\textbf{\LanguageName}&\textbf{\#}&\textbf{Glyphs}&\textbf{\LanguageName}&\textbf{\#}&\textbf{Glyphs}&\textbf{\LanguageName}\\\hline
0& &a&10&a &k&20  &l&u\\\hline
1&a&b&11&aa&l&30  &m&v\\\hline
2&b&c&12&ab&m&40  &n&w\\\hline
3&c&d&13&ac&n&50  &o&x\\\hline
4&d&e&14&ad&o&60  &p&y\\\hline
5&e&f&15&ae&p&70  &q&z\\\hline
6&f&g&16&af&q&80  &r& \\\hline
7&g&h&17&ag&r&90  &s& \\\hline
8&h&i&18&ah&s&100 &t& \\\hline
9&i&j&19&ai&t&1000&u& \\\hline
\end{tabular}
\end{center}

