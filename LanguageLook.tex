% Based on a Template which is
%     Creative Commons Attribution-ShareAlike 4.0 International
%     (CC BY-SA 4.0)
%     https://creativecommons.org/licenses/by-sa/4.0/

\chapter{The Look of \LanguageName}

\section{Transcription}

The symbols listed in the tables of the previous chapter are phonetic symbols.
These will be used to \emph{transcribe} \LanguageName\ words, but not to \emph{write} them.
To write them, we utilize a romanization system that should make the pronunciation fairly transparent.
That transcription system is listed below:

\begin{itemize}
\item
The following sounds will be written using the same letter as their phonetic symbol:
\textbf{m},
\textbf{n},
\textbf{p},
\textbf{b},
\textbf{t},
\textbf{d},
\textbf{k},
\textbf{g},
\textbf{q},
\textbf{ts},
\textbf{dz},
\textbf{f},
\textbf{v},
\textbf{s},
\textbf{z},
\textbf{h},
\textbf{l},
\textbf{w},
\textbf{i},
\textbf{u},
\textbf{e},
\textbf{o},
\textbf{ǝ},
and \textbf{a}.
\item
The sounds [tʃ] (kind of like the ``ch'' in ``\uline{ch}arge'') will be spelled \textbf{ch}.
\item
The sounds [dʒ] (kind of like the ``j'' in ``\uline{j}ar'') will be spelled \textbf{j}.
\item
The sound [ʝ] (similar to the ``z'' in ``a\uline{z}ure'') will be spelled \textbf{zh}.
\item
The sound [ɖ] (no English equivalent) will be spelled \textbf{d}.
\item
The sounds [ɟ] and [ʝ] (kind of like the ``j'' in ``\uline{j}ar'') will be spelled \textbf{j}.
\item
The sound [j] (like the ``y'' in ``\uline{y}ellow'') will be spelled \textbf{y}.
\item
The sound [ç] (like the ``h'' in ``\uline{h}eat'') will be spelled \textbf{hy}.
\item
The sound [ʍ] (like the ``wh'' in old pronunciations of ``\uline{wh}ich'') will be spelled \textbf{hw}.
\item
The sound [ŋ] (like the ``ng'' in ``si\uline{ng}'') will be spelled \textbf{ng}.
\item
The sound [ɲ] (like the ``ni'' in ``o\uline{ni}on'') will be spelled \textbf{ny}.
\item
The sound [ɴ] (no English equivalent) will be spelled \textbf{n} when occurring before uvular consonants.
\item
The sound [θ] (like the ``th'' in ``\uline{th}in'') will be spelled \textbf{th}.
\item
The sound [ð] (like the ``th'' in ``\uline{th}is'') will be spelled \textbf{dh}.
\item
The sound [ʃ] (like the ``sh'' in ``\uline{sh}e'') will be spelled \textbf{sh}.
\item
The sound [ʒ] (like the ``z'' in ``a\uline{z}ure'') will be spelled \textbf{zh}.
\item
The sounds [ɣ] and [ʁ] (no English equivalent) will be spelled \textbf{gh}.
\item
The sound [ɢ] (no equivalent in any well-known languages) will be spelled \textbf{qg}.
\item
The sound [ħ] (no English equivalent; sounds a bit like fogging up a mirror) will be spelled \textbf{h}.
\item
The sounds [x] and [χ] (like the ``ch'' Scottish ``lo\uline{ch}'') will be spelled \textbf{kh}.
\item
The sound [ɾ] (like the ``t'' or ``d'' in ``ma\uline{t}a\uline{d}or'') will be spelled \textbf{r}.
\item
The sound [r] (like the ``rr'' in Spanish ``pe\uline{rr}o'') will be spelled \textbf{r}.
\item
The sound [ʔ] (like the ``-'' in ``uh\uline{-}oh'') will be spelled \textbf{'}.
\item
The sound [ʕ] (no English equivalent; sounds a bit like gagging) will be spelled \textbf{'} (i.e. with an apostrophe).
\item
The sound [ɑ] (like the ``a'' in ``f\uline{a}ther'') will be spelled \textbf{a}.
(\emph{Note:} For the sake of simplicity, the sound [ɑ] will be transcribed [a] in the phonetic transcriptions below.)
\item
The sound [ɛ] (like the ``e'' in ``g\uline{e}t'') will be spelled \textbf{e}.
(\emph{Note:} For the sake of simplicity, the sound [ɛ] will be transcribed [e] in the phonetic transcription given in the relevant entries below.)
\item
The sound [e] (like the ``e'' in ``h\uline{e}y'') will be spelled \textbf{ei}.
\item
The sound [æ] (like the ``a'' in ``b\uline{a}d'') will be spelled \textbf{a}.
\item
The sound [ɪ] (like the ``i'' in ``k\uline{i}d'') will be spelled \textbf{i}.
\item
The sound [ɨ] (like the ``e'' in ``ch\uline{i}cken'') will be spelled \textbf{i}.
\item
The sound [ɔ] (like the ``aw'' in ``l\uline{aw}'') will be spelled \textbf{o}.
\item
The sound [œ] (no English equivalent; like the ``ö'' in German ``hören'') will be spelled \textbf{ö}.
(\emph{Note:} For the sake of simplicity, the sound [œ] will be transcribed [ø] in the phonetic transcription given in the relevant entries below.)
\item
The sound [y] (no English equivalent; like the ``ü'' in German ``für'') will be spelled \textbf{ü}.
\item
The sound [ɤ] (like the ``o'' in ``st\uline{o}ke'', but with the lips left unrounded) will be spelled \textbf{ë}.
\item
The sound [o] (like the ``o'' in ``r\uline{o}te'') will be spelled \textbf{ou}.
\item
The sound [ʊ] (like the ``oo'' in ``w\uline{oo}d'') will be spelled \textbf{u}.
\item
The sound [ɯ] (like the ``u'' in ``r\uline{u}ne'', but with the lips left unrounded) will be spelled \textbf{ï}.
\item
The sound [ǝ] (like the ``a'' in ``sof\uline{a}'') will be spelled \textbf{a}.
Long vowels will be written with a doubled version of the vowel (so [iː] will be written \textbf{ii}).
\end{itemize}

\section{Romanization}

This is the romanization system, which will be used to spell \LanguageName\ using the Roman alphabet.
This is also how to find \LanguageName\ words in the \LanguageName$\rightarrow$English dictionary.
The full system is described in detail below:

\begin{itemize}
\item
\textbf{A}, \textbf{a}:
Pronounced like the ``\uline{a}'' in ``f\uline{a}ther''.
\item
\textbf{Aa}, \textbf{aa}:
Pronounced like the ``\uline{a}'' in ``f\uline{a}ther'', but held slightly longer.
\item
\textbf{B}, \textbf{b}:
Pronounced like the ``\uline{b}'' in ``\uline{b}ad''.
\item
\textbf{Ch}, \textbf{ch}:
Pronounced like the ``\uline{ch}'' in ``ea\uline{ch}''.
Unlike the sound ``\uline{ch}'' in English ``\uline{ch}air'', there is no discernible puff of air that accompanies this sound.
If one holds one's breath while pronouncing the ``\uline{ch}'' in English ``\uline{ch}air'', one will pronounce this sound correctly.
\item
\textbf{D}, \textbf{d}:
Pronounced like the ``\uline{d}'' in ``\uline{d}iet''.
\item
\textbf{Dz}, \textbf{dz}:
Pronounced like the ``\uline{ds}'' in ``mo\uline{ds}''.
\item
\textbf{E}, \textbf{e}:
Pronounced like the ``\uline{e}'' in ``g\uline{e}t''.
\item
\textbf{Ǝ}, \textbf{ǝ}:
Pronounced like the ``\uline{a}'' in ``sof\uline{a}''.
\item
\textbf{F}, \textbf{f}:
Pronounced like the ``\uline{f}'' in ``\uline{f}og''.
\item
\textbf{G}, \textbf{g}:
Pronounced like the ``\uline{g}'' in ``\uline{g}oat'' (never like the ``\uline{g}'' in ``\uline{g}enius'').
\item
\textbf{Gh}, \textbf{gh}:
Pronounced like the ``\uline{r}'' in French ``\uline{r}ouge'' (never like the ``\uline{gh}'' in ``\uline{gh}ost'').
\textbf{H}, \textbf{h}:
Pronounced like the ``\uline{h}'' in ``\uline{h}at''.
If this letter is preceded by another letter that may imply a different pronounciation a small dot ($\cdot$) will be placed between the letters to indicate two separate consonants.
\item
\textbf{I}, \textbf{i}:
Pronounced like the ``\uline{i}'' in ``mach\uline{i}ne''.
\item
\textbf{Ii}, \textbf{ii}:
Pronounced like the ``\uline{i}'' in ``mach\uline{i}ne'', but held slightly longer.
\item
\textbf{J}, \textbf{j}:
Pronounced like the ``\uline{j}'' in ``\uline{j}am''.
\item
\textbf{K}, \textbf{k}:
Pronounced like the ``\uline{k}'' in ``s\uline{k}y'' (this sound features no aspiration.
Aspiration is the puff of air that occurs in the ``\uline{k}'' in ``\uline{k}ite''.
Compare the ``\uline{k}'' in ``\uline{k}ite'' and the ``\uline{k}'' in ``s\uline{k}y'' [try holding your hand in front of your face when pronouncing both].
The \LanguageName\ \textbf{k} should always be pronounced like the ``\uline{k}'' in ``s\uline{k}y'';
never like the ``\uline{k}'' in ``\uline{k}ite'').
\item
\textbf{K'}, \textbf{k'}:
There's no English equivalent to this sound.
This is an ejective consonant.
In the case of \textbf{k'}, it's pronounced just like \textbf{k}, but with one's breath held.
The result is a little ``popping'' sound that immediately follows the production of the \textbf{k}.
You can think of it as a \textbf{k} that's followed by a glottal \textbf{'} sound.
Producing those two sounds in short succession will result in a sound very close to \textbf{k'}.
Continue to practice and you should be able to get it.
\item
\textbf{Kh}, \textbf{kh}:
Pronounced like the ``\uline{ch}'' in the German pronunciation of ``Bu\uline{ch}''.
In English, this sound is commonly used with onomatopoeic words associated with disgust, like ``ble\uline{ch}!'' or ``i\uline{ch}!''
To pronounce it correctly, put your tongue in position to pronounce a \textbf{k}, but release it slowly;
allow the air to pass through the constricted space.
The result should be a sound like white noise.
\item
\textbf{L}, \textbf{l}:
Pronounced like the ``\uline{l}'' in ``\uline{l}ove''.
\item
\textbf{M}, \textbf{m}:
Pronounced like the ``\uline{m}'' in ``\uline{m}atter''.
\item
\textbf{N}, \textbf{n}:
Pronounced like the ``\uline{n}'' in ``\uline{n}ever''.
\item
\textbf{Ng}, \textbf{ng}:
Pronounced like the ``\uline{ng}'' in ``si\uline{ng}''.
\item
\textbf{O}, \textbf{o}:
Pronounced like the ``\uline{o}'' in ``t\uline{o}te''.
\item
\textbf{Ö}, \textbf{ö}:
Pronounced like the ``\uline{œu}'' in French ``s\uline{œu}r'', or the ``\uline{ö}'' in German ``h\uline{ö}ren''.
\item
\textbf{P}, \textbf{p}:
Pronounced like the ``\uline{p}'' in ``s\uline{p}ike'' (this sound features no aspiration.
Aspiration is the puff of air that occurs in the ``\uline{p}'' in ``\uline{p}ike''.
Compare the ``\uline{p}'' in ``\uline{p}ike'' and the ``\uline{p}'' in ``s\uline{p}ike'' [try holding your hand in front of your face when pronouncing both].
The \LanguageName\ \textbf{p} should always be pronounced like the ``\uline{p}'' in ``s\uline{p}ike'';
never like the ``\uline{p}'' in ``\uline{p}ike'').
\item
\textbf{Q}, \textbf{q}:
This is likely the most difficult sound in \LanguageName\ for an English speaker to master.
The sound is produced by touching the back of the tongue to the uvula and making a constriction as one would for a \textbf{k}.
One pronounces this sound like any other stop (\textbf{p}, \textbf{t}, \textbf{k}), it's just pronounced further back in the mouth than an English speaker is used to.
Think about when the doctor asks you to go, ``Ahhhhhhh$\ldots$''
Try doing that, and as you're doing it, take the back of your tongue, without moving it, and plug up the opening in the back of your mouth.
That should put you in perfect position to pronounce \textbf{q}.
\item
\textbf{R}, \textbf{r}:
Pronounced like the ``\uline{r}'' in Spanish ``pe\uline{r}o''.
Nearly identical to the ``\uline{t}'' or ``\uline{d}'' sound in English ``ma\uline{t}a\uline{d}or'' (pronounced quickly).
\item
\textbf{S}, \textbf{s}:
Pronounced like the ``\uline{s}'' in ``\uline{s}ad''.
\item
\textbf{Sh}, \textbf{sh}:
Pronounced like the ``\uline{sh}'' in ``\uline{sh}ade''.
\item
\textbf{T}, \textbf{t}:
Pronounced like the ``\uline{t}'' in ``s\uline{t}ake'' (this sound features no aspiration.
Aspiration is the puff of air that occurs in the ``\uline{t}'' in ``\uline{t}ake''.
Compare the ``\uline{t}'' in ``\uline{t}ake'' and the ``\uline{t}'' in ``s\uline{t}ake'' [try holding your hand in front of your face when pronouncing both].
The \LanguageName\ \textbf{t} should always be pronounced like the ``\uline{t}'' in ``s\uline{t}ake'';
never like the ``\uline{t}'' in ``\uline{t}ake'').
\item
\textbf{Ts}, \textbf{ts}:
Pronounced like the ``\uline{ts}'' in ``cu\uline{ts}''.
\item
\textbf{U}, \textbf{u}:
Pronounced like the ``\uline{u}'' in ``r\uline{u}minate''.
\item
\textbf{Uu}, \textbf{uu}:
Pronounced like the ``\uline{u}'' in ``r\uline{u}minate'', but held slightly longer.
\item
\textbf{Ü}, \textbf{ü}:
Pronounced like the ``\uline{u}'' in French ``r\uline{u}e'', or the ``\uline{ü}'' in German ``f\uline{ü}r''.
\item
\textbf{Üü}, \textbf{üü}:
Pronounced like the ``\uline{u}'' in French ``r\uline{u}e'', or the ``\uline{ü}'' in German ``f\uline{ü}r'', but held slightly longer.
\item
\textbf{V}, \textbf{v}:
Pronounced like the ``\uline{v}'' in ``\uline{v}an''.
\item
\textbf{W}, \textbf{w}:
Pronounced like the ``\uline{w}'' in ``\uline{w}alk''.
\item
\textbf{Y}, \textbf{y}:
Pronounced like the ``\uline{y}'' in ``\uline{y}et''.
\item
\textbf{Z}, \textbf{z}:
Pronounced like the ``\uline{z}'' in ``\uline{z}ebra''.
\item
\textbf{Zh}, \textbf{zh}:
Pronounced like the ``\uline{z}'' n ``a\uline{z}ure''.
\item
\textbf{'}:
This is referred to as a glottal stop, and is pronounced just like the catch in one's throat that occurs in between the ``uh'' and ``oh'' in English ``uh\uline{-}oh''.
This isn't a difficult sound to produce;
it just requires a bit of practice to insert it into words.
It will occur naturally in a string of vowels pronounced separately in English (e.g. if one were to say ``A A A A A A A'' [saying the actual name of the letter each time] over and over, a glottal stop will naturally occur before each instance of the vowel).
If one simply stops pronouncing a word mid-vowel and starts again, it will naturally occur.
(Note: It is important to remember that this apostrophe is not a stray mark, and not simply there for decoration.
The apostrophe stands for a consonant which has the same status as \textbf{g} or \textbf{k} or any other consonant.)
\item
\textbf{One Important Note Regarding Double Consonants:}
If a digraph (e.g. \textbf{kh}, \textbf{gh}, etc.) is doubled, only the first letter will be doubled (hence, \textbf{kkh} not \textbf{khkh}).
The consonant is pronounced like a doubled consonant, though, as actual combinations such as \textbf{k} followed by \textbf{kh} are impossible.
\end{itemize}

\section{Orthography}

\begin{center}
\begin{tabular}{|p{0.7\textwidth}|}
\hline
This section is only included if you are providing either vector images or a font.
Otherwise you need to say something along the lines of ``\LanguageName\ has a unique orthography used to write it but we will continue to use the romanization system as described above for ease and simplicity.''
\\\hline
\end{tabular}
\end{center}

\LanguageName\ has a unique orthography used to write it.
The orthography of \LanguageName\ consists of an alphabet/abjad/abudiga/syllabary/logogram/mixed system with two/three cases.
These two cases are called \textit{maiusklo} and \textit{minusklo} but they correspond to our upper case and lower case letters.
The upper case is used for names, places, and nouns otherwise the lower case is used.

Symbols are written left to right and lines move from the top of the page to the bottom.
When written right to left the shapes will be mirrored.
It is understood to consist of twenty-eight letters (called \emph{literoi}), though it really consists of twenty-two letters and two diacritical marks added to a small sub-set of the twenty-two letters.

\begin{itemize}
\item \textbf{a} is indicated by the upper case symbol {\confont A} or the lower case symbol {\confont a} which sometimes has the alternative form of \textsl{a}
\item \textbf{b}  is indicated by the upper case symbol {\confont B}  or the lower case symbol {\confont b}
\item \textbf{ch} is indicated by the upper case symbol {\confont Cx} or the lower case symbol {\confont cx}
\item \textbf{d}  is indicated by the upper case symbol {\confont D}  or the lower case symbol {\confont d}
\item \textbf{dz} is indicated by the digraph {\confont dz}
\item \textbf{e}  is indicated by the upper case symbol {\confont E}  or the lower case symbol {\confont e}
\item \textbf{f}  is indicated by the upper case symbol {\confont f}  or the lower case symbol {\confont f}
\item \textbf{g} is indicated by the upper case symbol {\confont G} or the lower case symbol \textsf{g}, but \LanguageName\ has a long tradition of printing which uses the lower case symbol {\confont g} and most native charts will show {\confont g} even most people will write \textsf{g}
\item \textbf{h}  is indicated by the upper case symbol {\confont H}  or the lower case symbol {\confont h}
\item \textbf{i} following \textbf{a} or \textbf{o} (both dipthongs) is indicated by the lower {\confont j}, but in all other cases is indicated by the upper case symbol {\confont I} or the lower case symbol {\confont i}
\item \textbf{k}  is indicated by the upper case symbol {\confont K}  or the lower case symbol {\confont k}
\item \textbf{kh} is indicated by the upper case symbol {\confont Hx} or the lower case symbol {\confont hx}
\item \textbf{l}  is indicated by the upper case symbol {\confont L}  or the lower case symbol {\confont l}
\item \textbf{m}  is indicated by the upper case symbol {\confont M}  or the lower case symbol {\confont m}
\item \textbf{n}  is indicated by the upper case symbol {\confont N}  or the lower case symbol {\confont n}
\item \textbf{o}  is indicated by the upper case symbol {\confont O}  or the lower case symbol {\confont o}
\item \textbf{p}  is indicated by the upper case symbol {\confont P}  or the lower case symbol {\confont p}
\item \textbf{r}  is indicated by the upper case symbol {\confont R}  or the lower case symbol {\confont r}
\item \textbf{s}  is indicated by the upper case symbol {\confont S}  or the lower case symbol {\confont s}
\item \textbf{sh} is indicated by the upper case symbol {\confont Sx} or the lower case symbol {\confont sx}
\item \textbf{t}  is indicated by the upper case symbol {\confont T}  or the lower case symbol {\confont t}
\item \textbf{ts} is indicated by the upper case symbol {\confont C}  or the lower case symbol {\confont c}
\item \textbf{u} following \textbf{a}, \textbf{e}, \textbf{o} (dipthongs) is indicated by the upper case symbol {\confont Ux} or the lower case symbol {\confont ux}, but in all other cases is indicated by the upper case symbol {\confont U} or the lower case symbol {\confont u}
\item \textbf{v}  is indicated by the upper case symbol {\confont V}  or the lower case symbol {\confont v}
\item \textbf{y}  is indicated by the upper case symbol {\confont J}  or the lower case symbol {\confont j}
\item \textbf{z}  is indicated by the upper case symbol {\confont Z}  or the lower case symbol {\confont z}
\item \textbf{zh} is indicated by the upper case symbol {\confont Jx} or the lower case symbol {\confont jx}
\end{itemize}

A basic pronunciation chart would look like:

\begin{center}
\scriptsize
\begin{tabular}{|c|c|c|c|c|c|c|}\hline
\rule{0pt}{7mm}{\LARGE\confont A a}&{\LARGE\confont B  b} &{\LARGE\confont C c}&{\LARGE\confont Cx cx}&{\LARGE\confont D  d} &{\LARGE\confont E e}&{\LARGE\confont F  f} \\
{\bf a}&{\bf b}&{\bf ts}&{\bf ch}&{\bf d}&{\bf e}&{\bf f}\\\hline
\rule{0pt}{7mm}{\LARGE\confont G g}&{\LARGE\confont Gx gx}&{\LARGE\confont H h}&{\LARGE\confont Hx hx}&{\LARGE\confont I  i} &{\LARGE\confont J j}&{\LARGE\confont Jx jx}\\
{\bf g}&{\bf gh}&{\bf h}&{\bf kh}&{\bf i}&\textbf{y} or \textbf{i}&{\bf zh}\\\hline
\rule{0pt}{7mm}{\LARGE\confont K k}&{\LARGE\confont L  l} &{\LARGE\confont M m}&{\LARGE\confont N  n} &{\LARGE\confont O  o} &{\LARGE\confont P p}&{\LARGE\confont R  r} \\
{\bf k}&{\bf l}&{\bf m}&{\bf n}&{\bf o}&{\bf p}&{\bf r}\\\hline
\rule{0pt}{7mm}{\LARGE\confont S s}&{\LARGE\confont Sx sx}&{\LARGE\confont T t}&{\LARGE\confont U  u} &{\LARGE\confont Ux ux}&{\LARGE\confont V V}&{\LARGE\confont Z  z} \\
{\bf s}&{\bf sh}&{\bf t}&{\bf u}&{\bf u}&{\bf v}&{\bf z}\\\hline
\end{tabular}
\end{center}

\section{Typing \LanguageName}

\begin{center}
\begin{tabular}{|p{0.7\textwidth}|}
\hline
This section is only included if you are providing a font.
In case you are curious, \texttt{\LanguageFont} is a cut-down and slightly modified \texttt{FreeSerif.ttf}.
\\\hline
\end{tabular}
\end{center}

The font file is called \texttt{\LanguageFont}.
Before you can use it, you will need to install it.
In almost all modern operating systems, installation consists of double-clicking on the file, seeing a preview of the font, and clicking on a button with the word ``Install'' on it.
Once you have accomplished that, close out of all your word processors and restart them to ensure that they know the font is available.
You should now see a font named \LanguageName.

The correspondence between the romanization and the sounds of \LanguageName\ has influenced the method of typing in this alphabet.
The symbol of the left is the symbol typed if you hold the shift key down while pressing that key and the symbol on the right is the symbol typed if you do not hold the shift key down.
If we consider the typical qwerty keyboard, then the layout is as follows:

\begin{center}
\scriptsize
\begin{tabular}{cccccccccccccccccccc}
\hline
\multicolumn{2}{|c}{\rule{0pt}{4mm}\Large Q}&
\multicolumn{2}{|c}{\Large W}&
\multicolumn{2}{|c}{\Large E}&
\multicolumn{2}{|c}{\Large R}&
\multicolumn{2}{|c}{\Large T}&
\multicolumn{2}{|c}{\Large Y}&
\multicolumn{2}{|c}{\Large U}&
\multicolumn{2}{|c}{\Large I}&
\multicolumn{2}{|c}{\Large O}&
\multicolumn{2}{|c|}{\Large P}\\
\multicolumn{2}{|c}{}&
\multicolumn{2}{|c}{}&
\multicolumn{2}{|c}{\confont E e}&
\multicolumn{2}{|c}{\confont R r}&
\multicolumn{2}{|c}{\confont T t}&
\multicolumn{2}{|c}{}&
\multicolumn{2}{|c}{\confont U u}&
\multicolumn{2}{|c}{\confont I i}&
\multicolumn{2}{|c}{\confont O o}&
\multicolumn{2}{|c|}{\confont P p}\\\hline
&
\multicolumn{2}{|c}{\rule{0pt}{4mm}\Large A}&
\multicolumn{2}{|c}{\Large S}&
\multicolumn{2}{|c}{\Large D}&
\multicolumn{2}{|c}{\Large F}&
\multicolumn{2}{|c}{\Large G}&
\multicolumn{2}{|c}{\Large H}&
\multicolumn{2}{|c}{\Large J}&
\multicolumn{2}{|c}{\Large K}&
\multicolumn{2}{|c|}{\Large L}\\&
\multicolumn{2}{|c}{\confont A a}&
\multicolumn{2}{|c}{\confont S s}&
\multicolumn{2}{|c}{\confont D d}&
\multicolumn{2}{|c}{\confont F f}&
\multicolumn{2}{|c}{\confont G g}&
\multicolumn{2}{|c}{\confont H h}&
\multicolumn{2}{|c}{\confont J j}&
\multicolumn{2}{|c}{\confont K k}&
\multicolumn{2}{|c|}{\confont L l}\\\cline{2-19}
&&
\multicolumn{2}{|c}{\rule{0pt}{4mm}\Large Z}&
\multicolumn{2}{|c}{\Large X}&
\multicolumn{2}{|c}{\Large C}&
\multicolumn{2}{|c}{\Large V}&
\multicolumn{2}{|c}{\Large B}&
\multicolumn{2}{|c}{\Large N}&
\multicolumn{2}{|c|}{\Large M}\\&&
\multicolumn{2}{|c}{\confont Z z}&
\multicolumn{2}{|c}{accents}&
\multicolumn{2}{|c}{\confont C c}&
\multicolumn{2}{|c}{\confont V v}&
\multicolumn{2}{|c}{\confont B b}&
\multicolumn{2}{|c}{\confont N n}&
\multicolumn{2}{|c|}{\confont M m}\\\cline{3-16}
\end{tabular}
\end{center}

The lowercase \textbf{x} key is used to add a compatible accent the the base character.
Uppercase \textbf{X} will not work.
If you want to see either a {\confont Cx} or a {\confont Ux} you need to type \texttt{Cx} or \texttt{Ux}.

Many spellings are based on historic pronunciations or borrowings and will not follow the general rules.
The dictionary will show the current spelling, the way to type it using this font, and the modern pronounciation.

