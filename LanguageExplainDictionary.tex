% Based on a Template which is
%     Creative Commons Attribution-ShareAlike 4.0 International
%     (CC BY-SA 4.0)
%     https://creativecommons.org/licenses/by-sa/4.0/

\chapter{How to Read the Dictionary}

\begin{center}
\begin{tabular}{|p{0.7\textwidth}|}\hline
The fourth item explained here should only be present if you are also providing a font.
The fifth item explained here should only be present if you are also providing vector drawings or a font.
\\\hline
\end{tabular}
\end{center}

Below is a sample dictionary entry from the \LanguageName$\rightarrow$English side of the dictionary.
Each item has a number which corresponds to an explanation below the entry:

{\ }

\noindent
\textbf{wordi}$^1$
[word.'i]$^2$
*wrd$^3$
\texttt{cxhxsxux}$^4$
{\confont cxhxsxux}$^5$
\{24\}$^6$
(v.)$^7$
to gloss$^8$
(nice note)$^9$

\begin{enumerate}
\item
\textbf{Citation Form:}
\LanguageName\ words will be listed by the citation form.
This is the basic romanized form that will be used in scripts and translations.
When preparing artwork, the Orthographic Form should be used, in which case the Typographic Form should be followed closely to produce an accurate result.
\item
\textbf{Phonetic Form:}
This is how the word is pronounced.
The transcription will be written in \textsc{ipa}, and will be more or less phonetic (unimportant details being left out).
You'll notice in this example that the citation form and the phonetic form do match fairly closely, as intended by the romanization system.
\item
\textbf{Etymology:}
This is the etymology for each word.
It will prove crucial in derived languages.
In \LanguageName\ it's more or less a repetition of the information already provided.
\item
\textbf{Typographic Form:}
This is \emph{exactly} what you type to produce the \LanguageName\ forms in the Orthographic Form using the provided font.
\item
\textbf{Orthographic Form:}
This is how the word will look in the orthography of the native \LanguageName\ writing system.
\item
\textbf{Word Type:}
This system identifies how common a given word is and whether or not it's impolite---plus it identifies words that may be interesting examples to share.
The system is as follows:
\begin{itemize}
\item
\{X$\ldots$\}:
The first digit is a measure of how well known the word is.
\begin{enumerate}
\item[1]
indicates that this word is so basic to the language it is difficult to understand without these words.
For English this would be things like pronouns---even utterances like ``me Jane, you John'' will require an understanding of these.
For \LanguageName\ this also includes some prepositions.
\item[2]
indicates basic words that either affect meaning in ways not obvious or considered a basic part of interacting.
In this case wordi is the most basic way to ask for information, so it has been provided with this level.
\item[3]
indicates words that are common and likely to be used by all people in most situations.
If a native doesn't know this word, they can probably guess it anyway.
This is the default type.
\item[4]
indicates words that are either a less used synonym or words only used is special contexts like riddles or other wordplay.
May also be words that are no longer part of the language but used to be an can still be found in idioms like the English ``to and fro''.
\item[5]
indicates words that are technical but may be generally known.
If it is known the generally understood meaning may be different than the technical meaning like doctor's referring to a condition as ``chronically acute''.
\end{enumerate}
\item
\{$\ldots$X\}:
The second digit is a measure of how polite the word is.
\begin{enumerate}
\item[1]
indicates words that are considered cumpulsory for basic politeness.
If you fail to use these words you may offend.
\item[2]
indicates that this is considered a polite word to use.
In English ``please'' and ``thank you'' would be this level.
\item[3]
indicates words that are neutral and usable in any situation.
This is the default type.
\item[4]
indicates words that are crude but not necessarily offensive.
In some contexts these words may no longer be considered crude.
Sometimes that context is situational---like when visiting a specialist.
Sometimes it is based on relationships---some things you can say to close friends that may be rude otherwise.
In this case it is rude to ask for any information, but this is the only way to do it.
\item[5]
indicates words that are both crude and offensive.
It is not unusual to find someone appologizing for using these words.
\end{enumerate}
\item
\{$\ldots$\}*
An asterisk attached to the type indicates that the word is unique or interesting enough to pay attention to.
It may be a word that plays a crucial role in the lore of the world or that has to do with gameplay.
The asterisk following the curly bracket makes these words easy to search for.
\end{itemize}
\item
\textbf{Part of Speech:}
The part of speech refers to the \LanguageName\ part of speech, and not the English part of speech.
If a word fits more than one part of speech, it is listed more than once.
Our example is a verb.
Below is a list of the \LanguageName\ parts of speech:
\begin{itemize}
\item (adj.) = adjective
\item (adv.) = adverb
\item (cc.) = coordinating conjunction	
\item (cir.) = circumposition	
\item (conj.) = conjunction	
\item (cor.) = correlative (equivalent to the Englsh WH-words)
\item (def.) = definite article
\item (det.) = determiner	
\item (expr.) = expression	
\item (n.) = noun
\item (nm.) = name	
\item (np.) = noun phrase	
\item (num.) = number
\item (part.) = particle	
\item (pl.) = plural	
\item (post.) = postposition	
\item (pref.) = prefix	
\item (prep.) = preposition
\item (pron.) = pronoun	
\item (sc.) = subordinating conjunction	
\item (sg.) = singular	
\item s/o = someone	
\item s/t = something	
\item (suf.) = suffix	
\item s/w = somewhere	
\item (v.) = verb	
\item (v3.) = tripartate verb
\item (vi.) = intransitive verb	
\item (vt.) = transitive verb	
\item (vp.) = verb phrase	
\end{itemize}
\item
\textbf{Definition:}
Some words will have single word definitions, others a semantic description, and some will have multiple definitions.
In this case we often provide some semantic conditions to a single word definition.
\item
\textbf{Extra Information:}
Extra information will appear in parentheses.
The extra information is there to give the reader a clearer idea of exactly how the word is used.
Often socio-cultural information about a given word will appear in parentheses.
In this case since the scope has a very small semantic context than ask.
\end{enumerate}

The English$\rightarrow$\LanguageName\ side of the dictionary is less of a lexicon and more of a glossary.
It will attempt to provide the reader with a one-to-one translation of a given word.
The parts of speech should be familiar as they will be English parts of speech.

