% Based on a Template which is
%     Creative Commons Attribution-ShareAlike 4.0 International
%     (CC BY-SA 4.0)
%     https://creativecommons.org/licenses/by-sa/4.0/

\chapter{\LanguageName\ Description}

\section{Morphology and Syntax}

\LanguageName\ is full of inflections that permit some freedom in word order but the default structure of \LanguageName\ is fairly regular.

\LanguageName\ is a head final, subject-verb-object, agglutinating language.
Genitives follow nouns and the word \emph{de},
adpositions are called prepositions because they precede their clauses,
and nouns must precede their relative clauses.
Finally, even though \LanguageName\ is head-final, adverbs follow their verbs.

\begin{center}
\begin{tabular}{l}
The red bird quickly went to the dog's house.\\
\ \ \ \ \ would be\\
The red bird went quickly to house of the dog.\\
\end{tabular}
\end{center}

\begin{center}
\begin{tabular}{|p{0.7\textwidth}|}
\hline
If you are doing a language sketch, then this is the end of the chapter and no other information is provided.
\\\hline
\end{tabular}
\end{center}

\section{Nouns}

\LanguageName\ nouns inflect for number, case, gender, possesion, $\ldots$

Noun cases are: actor, patient, or broker and these functions/cases are identified by adding/changing/removing $\ldots$

Noun numbers are: singular, dual, or plural.

Noun genders are: animate or inanimate.

\section{Adjectives}

\LanguageName\ adjectives (dis)agree with their nouns but only in case and number.
They are further inflected for evidence.

Because of the agreement with their nouns, adjectives can come before or after but they are most often placed before.
Adjectives must come before the nouns they modify.
If there are multiple adjectives they are placed in order of place, time, and color.

Adjective numbers are: singular and plural.
If the attached noun is dual, the adjective is marked for singular.

Adjective evidences are: seen and heard.

\section{Demonstratives}

All \LanguageName\ demonstratives are deictic and are inflected only for gender.

\begin{quote}
\textit{tie} means that place (there) and is inherently inanimate.\\
\textit{tiu} means that animate thing.\\
\textit{tius} means that inanimate thing.\\
\end{quote}

\section{Verbs}

\LanguageName\ verbs conjugate for tense and believability and must (dis)agree with their adverbs.
Copulas are formed by using an adjective as the object of the present tense of the verb \textit{esti};
\textit{la birdo estas ruja} means ``the bird is blue''.
Negation is handled by replacing all the vowels with \textbf{a} so \textit{la birdo ne astas ruja} means ``the bird isn't blue''.

The tenses are: past, present, future, and unlikely.
Past tense is created by adding \textbf{t} before the verb.

\section{Adverbs}

Adverbs must follow the verb they modify and are of three types: manner, location, and time.

Manner adverbs are formed by taking an adjective and inflecting it with a following \textbf{e}:
\textit{mi iris rapide} ``I went quickly''

Time adverbs are unique words:

\begin{quote}
\textbf{a} means long time ago.\\
\end{quote}

\begin{center}
\begin{tabular}{|p{0.7\textwidth}|}
\hline
Unless you are doing an Advanced Grammar, the chapter is almost over.
A Basic Grammar should skip the next several sections and finish the chapter with the historical notes.
\\\hline
\end{tabular}
\end{center}

\section{Participles}

Participles come in two basic forms:
active, meaning that the subject is doing the verb that the participle is based on;
passive, meaning that the subject is having the verb done to them that the participle is based on.

Both types of participles inflect for tense and number:
\textit{vid\textbf{int}a} means ``was seeing'',
$\ldots$.

\section{Coordination}

Conjunctions are handled by using the preposition:
\textit{ka} means and,
$\ldots$

\section{Relative Clauses}

A relative clause as the subject//noun is formed by $\ldots$.-

Direct object/indirect object/adposition/possessive.

\section{Questions}

Yes/no questions, that expect the answer to be agreement, are formed by taking a statement and changing $\ldots$

Yes/no questions, that expect the answer to be dis-agreement, are formed by taking a statement and $\ldots$

Questions asking for information are formed by taking a statemend and replacing the subject/object with one of the following words:
\textit{it} meaning a place ``where'',
$\ldots$

\section{Historical Notes}

\begin{center}
\begin{tabular}{|p{0.7\textwidth}|}
\hline
If you are making a Basic Grammar, you should include this section.
It shouldn't need to be said, but an Advanced Grammar should also include this section.
\\\hline
\end{tabular}
\end{center}

Below is a behind-the-scenes description of the historical processes that gave rise to the current alternations seen in \LanguageName.
In the descriptions below, a segment, word or phrase preceded by an asterisk (*) is a proto-form.
A proto-form is an older form that's no longer present in the modern language.
List the changes in order.

\begin{itemize}
\item
\textbf{Loss of Schwa:}
*ǝ > Ø / \_,G

The basic schwa was lost next to glides and the glottal stop.
These sounds affected a change in the surrounding vowels, resulting in modern i, u, and a, as well as older *ai and *au.
\item
\textbf{Loss of Diphthongs:}
*ai, *au > e, o

diphthongs *ai and *au became e and o, respectively.
\item
\textbf{Schwa Lowering:}
*ǝ > a / \{C[+back], G\}\_

Schwa lowered to a when it followed q,kh, gh, w, y or ‘.
\item
\textbf{Vowel Lowering:}
*i, *u > e, o / [+back]\_

Vowels lowered when they followed q,kh, gh, or ‘.
\item
\textbf{Stop Insertion:}
Ø > C[-cont] / N\_L

A homorganic stop is inserted inbetween a nasal and a liquid.
\item
\textbf{Nasal Assimilation:}
N > [αplace] / \_,C[αplace]

Nasals assimilate in place to a following or preceding consonant, with few exceptions.
One exception is the velar nasal, which doesn't assimilate in place to a following consonant unless that consonant is y.
Other exceptions will be noted when they occur.
\item
\textbf{Vowel Fronting:}
V[-low] > [+front] / V[+front]...\_\#

In many instances, the vowels u and o fronted to ü and ö respectively when occurring after the vowels i, ü, ö or e.
This only happened when there were no other intervening vowels or glottalic consonants (t', ts', k', q', ', *b', *d'), and preferentially in closed syllables.
Also, it only occurred with *e (not *ai).
\item
\textbf{Vowel Devoicing:}
V > [-voice] / C[+CG]\_C[-voice]

Vowels devoice in between ejectives and voicelesssounds.
\item
\textbf{Progressive Voicing Assimilation:}
C > [αvoice] / \_C[αvoice]

Generally a consonant takes on the voicing of the one followingit.
\item
\textbf{Back Vowel Lenition:}
V[+back, -low] > v / \_V

Where ordinarily vowel hiatus would result in either a diphthong or two vowel nuclei separated by a glottal stop, non-low back vowels instead become the semi-vowel/fricative v.
This occurs, for example, when the perfect prefix k'u- occurs directly before a vowel-initial verbal stem other than u or ü.
\item
\textbf{Loss of Implosives:}

All implosives becameand *d' > d.
simple voiced plosives in all environments: *b' > b,
\item
\textbf{Affricate Gemination:}

Sequences of affricates become a single affricate with a geminate onset: tsts > tts, dzdz > ddz, ts'ts' > tts'.
\item
\textbf{Word-Final Stop Simplification:}
*C[+stop] > [-voice, -CG] / \_\#

All stops became plain voiceless stops at the end of a word.
(Note that q' becomes k at the end of a word.)
\item
\textbf{Compensatory Lengthening:}
*V > Vː / \_C[+voice]

All vowels lengthened before word-final voiced obstruents that became voiceless as a result of the previous change.
Prominent vowels in diphthongs became long, destroying the on- and off-glide-like vowels in the process.
\item
\textbf{Glottalic Dissimilation:}
C > [-glottalic] / C[+glottalic]V\_

This rule prevents consecutive ejectives from occurring in the language.
\item
\textbf{Loss of Voiced Velar Stop:}
*g > ng

This was a ubiquitous sound change.
\item
\textbf{Loss of Long Mid Vowels:}
*ey, *ee > ii; *ow, *oo > uu

This was a ubiquitous sound change.
\item
\textbf{Diphthong Simplification:}
*aw > o; *ew > u; *ay > e; *oy > i / \_, Stress; *uw > uu; *iy > ii

A diphthong will become the corresponding monophthong when it is unstressed.
The latter two changes affecting a sequence of a high vowel followed by a glide occur in all instances.
\item
\textbf{Nasal Assimilation:}
C[+nasal] > [αplace] / \_C[αplace]

This happened with all nasals.
\item
\textbf{Fortition:}
V > Vː / \_k', ch', t', ', h, w, y; t, k, s, sh, kh, l, m, n, ng, ny, b, d, j > tt, kk, ss, ssh, kkh, ll, mm, nn, nng, nny, bb, dd, jj

Stressed syllables strengthen either by lengthening the vowel or doubling the coda consonant.
What happens with each specific coda consonant is shown above.
\item
\textbf{Lenition:}
*t', *ch', *k', *t, *ch, *k, *b, *d, *j, *kh, *h > t, ch, k, d, j, *g, m, n, ny, Ø, Ø

Lenition occurs outside of the first syllable at the head of a strong syllable (stressed CVC syllable).
To give an example, with an underlying form like /tak-u-n-s/, the result would be takkunaas.
With an underlying form like /tak-tak-u-n-s/, though, the result would be tattagunnas, with the underlying /k/ leniting to g (note: this sound later changed to ng, so the final form would be tattangunnas).
\item
\textbf{Devoicing:}
C[+obs] > [-voice] / \_C[-voice]

The voiced obstruents *z, *v, *b, *d and *g devoice to s, f, p, t and k respectively when occurring before s, h, p, t and k.
Also, *z devoices to s in word-final position.
\item
\textbf{Voicing:}
C > [+voice] / C[+nasal]\_, \_C[+voice, -cont]

The voiceless stops *p, *t and *k voice to b, d and g after nasals and before b, d and g.
\item
\textbf{Fortition I:}
C > [+stop] / C[+nasal]\_

The fricatives/approximants *v, *z, *l, *r and *s all become stops when occurring after nasals.
The fricative *v becomes b after nasals, and the other consonants become d.
(Important note: This sound change continues to occur, and is applied again after the last rule listed here.)
\item
\textbf{Fortition II:}
C[+lateral] > [+stop] / \_V[+high]

The approximants *l and *r become d when occurring directly before u or i.
Note that this does not occur if the vowel has resulted from a sequence whose first vowel wasn't high and which occurred as a result of glide simplification (see below).
\item
\textbf{Glottal Deletion I:}
C[+glottal] > Ø / V\_V[-back], \_\#

The voiceless fricative h disappears after a vowel and before i or e.
The h is retained if the vowels are identical.
The consonant h is always lost word-finally, though (occasionally reappearing if the word comes before another that begins with a vowel).
\item
\textbf{Glottal Deletion II:}
C[+glottal] > Ø / C[+obs]\_

The voiceless fricative h disappears after stops,fricatives and approximants.
\item
\textbf{Hiatus:}
VV > CV, V

The following happens when two vowels come into contact:
*aa, *ea, *oa > a; *ii, *ei > i; *uu, *eu, *ou > u; *au, *ao > o; *ai, *ae, *oi > e; *ia > ya; *ua > wa; *ie > ye; *ue > we; *io > yo; *uo > wo; *iu > yu; *ui > wi.
\item
\textbf{Glide Gemination:}
Cx[+glide]Cx[+glide] > [+continuant]

When two glides come into contact with one another, it produces the following results:
*ww > v; *yy > j; *hh > h.
Note that the latter rule applies even in situations where h would ordinarily be deleted (see above).
\end{itemize}

